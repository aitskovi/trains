\documentclass{article}

\title{CS 452 Kernel 2 Documentation}
\author{
  Avi Itskovich, 20332164
  \and
  Alex Ianus, 20342535
}

\begin{document}

\maketitle

\section{Overview}

This documentation covers the implementation of the Kernel for Assignment 2. This assignment covered message passing and the name server. A Rock/Paper/Scisssors game was implemented as a user service. The new system calls are: 
\begin{itemize}
  \item int Send(int priority, (void *code)())
  \item int Recieve()
  \item int Reply()
  \item int WhoIs(char *name)
  \item int RegisterAs(char *name)
\end{itemize}
As part of our work, we also cleaned up scheduling a bit to make it faster. The kernel runs the user task defined by the assignment, and provides the requested output.

\section{Operating Instructions}
\begin{enumerate}
  \item load -b 0x00218000 -h 10.15.167.4 "ARM/ai/kernel2.elf"
  \item go
  \item After each round of rock/paper/scissors, press any key to continue
\end{enumerate}

\section{Submitted Files}
\begin{itemize}
    \item
\end{itemize}

\section{Kernel Description}

\subsection{Messaging}

Messaging in our system is implemented in two parts. One part handles the actual movement of data between buffers and maintaining messaging ordering while the second part handles scheduling, and error verification. The code is in two seperate files:
\begin{itemize}
    \item messaging.{c/h} - Buffer management
    \item ksyscall.{c/h} - Scheduling
\end{itemize}
The reasoning behind is to seperate areas of concern. The buffer management layer returns appropriate error codes so scheduling layer can respond and change task states appropriately.

\subsubsection{Buffer Management}

In our messaging system each task has associated with it:
\begin{itemize}
    \item queue of tids - A queue of tids that have a message for you.
    \item char ** msg - A reference to the msg buffer that this task is sending.
    \item int msglen - The length of the msg buffer you're referring to.
    \item char ** rcv - A reference to the msg buffer you recieve data in.
    \item int rcvlen - The length of the rcv buffer.
    \item int *src - A reference to an integer which is filled with the src tid of the msg.
\end{itemize}

The cost of adding an extra task is then: 20 bytes + size of the queue. In the future we're going to replace the queue with a list of task pointers which should make our memory cost per task 28 bytes.

Some interesting parts of the system are our re-use of the rcv, rcvlen items for both sending and recieving. In the case that Recieve(), blocks we put the data for recieve into rcv, rcvlen and src. This allows us to later use these buffers when someone else sends to us. Similarly, a call to Send fills rcv, rcvlen, and src with the buffers for reply.

We know that only one of those can happen at a time, so there will be no overwriting conflicts.

\subsubsection{Task Scheduling}

To indicate blocked tasks we simply change their "task state" to an appropriate value. These values are determined based on the return codes from the buffer management system.

\subsection{NameServer}

Since we do not plan on supporting short-lived tasks, we decided to make the NameServer slower but simpler. Our NameServer simply stores an array of registrations. A registration is simply a tid -> name mapping. The WhoIs request will traverse this list, trying to find one where the name matches the one we requested. The tid of that registration will be returned if no matching registration is found, an error code is returned. RegisterAs, will similarly traverse the list until it either find a registration with a matching name which it will overwrite or an empty registration slot. If there are no empty slots an error will be returned.

We plan to either change RegisterAs to block in the future, or create a seperate polling primitive to simplify synchronizing task startup.

\subsection{Scheduling}

For kernel 2, once we started timing things we noticed that our old method of scheduling was unoptimal. We previously used a circular buffer as a queue and decided to switch to using linked list with the next pointers inside the task descriptors. The reasoning behind this is a decreased space requirement, as well as it being extendible to other parts of the system where we need lists of tasks or tids.

\subsection{Utilities}

As part of implementing different parts of the system we started having to add some basic utility functions. These include memcpy, memset, strlen, and streq. These are currently implemented in a simple way to save time. We will later optimize these as required.

\subsection{Timing}

Timing is done using the 40-bit timer running at 938khz to get maximum granularity in the timing results. Timing is accomplished by creating a timer struct, resetting it (which stored the current time), and then reading the elapsed time (by getting the current time and subtracting the stored time).

\subsection{Rock/Paper/Scissors}
The user programs used to test send/receive/reply are a RockPaperScissors server and a set of RockPaperScissors clients. 

\subsubsection{RPSServer}
The server first registers itself with the nameserver and then enters an infinite messaging handling loop.
If it receives a SIGNUP message it places the client task on a queue. Once there are two clients on the queue, they are popped off, turned into a match, and unblocked with a GOAHEAD reply.
If a PLAY message is received, the play is registered, and if both players in a match have played they are unblocked with a RESULT reply.
If a QUIT message is received, a "FORFEIT" play is registered and the quitting task is immediately unblocked. Their opponent is notified by receiving a "FORFEIT" result instead of WIN, LOSE, or DRAW.

\subsubsection{RPSClient}
Each of the six clients first looks up the tid of the server using the nameserver. It then signs up for a game, blocks until it has an opponent, plays two rounds (printing the results as they find out), and then QUITs and exits.

\subsection{Performance Measurements}
\begin{verbatim}
4 bytes		off	yes	off	06	528
64 bytes	off	yes	off	06	1065
4 bytes		on	yes	off	06	38
64 bytes	on	yes	off	06	73
4 bytes		off	no	off	06	532
64 bytes	off	no	off	06	1050
4 bytes		on	no	off	06	38
64 bytes	on	no	off	06	72
4 bytes		off	yes	on	06	261
64 bytes	off	yes	on	06	422
4 bytes		on	yes	on	06	18
64 bytes	on	yes	on	06	27
4 bytes		off	no	on	06	265
64 bytes	off	no	on	06	429
4 bytes		on	no	on	06	19
64 bytes	on	no	on	06	27
\end{verbatim}

We see a lot of time being spent in memcpy, as evidenced by large (~100\%) increases in latency when the message length is changed from 4 bytes to 64 bytes. We also see that memory accesses are taking the vast majority of the time, since turning caches on is good for >13x improvement in latency.

\subsection{Output}
\begin{verbatim}
RPS Client Task 3 starting
RPS Client Task 4 starting
RPS Client Task 5 starting
RPS Client Task 6 starting
RPS Client Task 7 starting
RPS Client Task 8 starting
RPS Client Task 3 signed up
RPS Client Task 4 signed up
RPS Client Task 5 signed up
RPS Client Task 6 signed up
RPS Client Task 7 signed up
RPS Client Task 8 signed up
Round ended with 3 playing Paper and 4 playing Rock
Press any key to continue 
I, task 3, won
I, task 4, lost
Round ended with 5 playing Paper and 6 playing Rock
Press any key to continue 
I, task 5, won
I, task 6, lost
Round ended with 7 playing Paper and 8 playing Rock
Press any key to continue 
I, task 7, won
I, task 8, lost
Round ended with 3 playing Paper and 4 playing Paper
Press any key to continue 
I, task 3, drew
I, task 4, drew
Round ended with 5 playing Paper and 6 playing Paper
Press any key to continue 
I, task 5, drew
I, task 6, drew
Round ended with 7 playing Rock and 8 playing Scissors
Press any key to continue 
I, task 7, won
I, task 8, lost
\end{verbatim}

The output shows the six client tasks starting up and signing up for a game. They each then play two rounds of rock paper scissors before exiting. The rock paper scissors server emits the result of each round after it's completed. Since the clients signed up in order (3, then 4, then 5, then 6, then 7, then 8) and since they are put on a queue when they sign up, we see client 3 playing with client 4, client 5 playing with client 6, and client 7 playing with client 8.

\subsection{Known Bugs and Limitations}
\begin{itemize}
    \item The timing will roll over if the OS is running for over 79 minutes since we only consider the low byte of the 40-bit timer value
\end{itemize}
\end{document}
