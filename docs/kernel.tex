\documentclass{article}

\title{CS 452 Kernel Documentation}
\author{
  Avi Itskovich, 20332164
  \and
  Alex Ianus, 20342535
}

\begin{document}

\maketitle

\section{Overview}

This documentation covers the implementation of the Kernel for Assignment 1. This includes basic scheduling, context switching and task creation. The provided system calls are:
\begin{itemize}
  \item int Create(int priority, (void *code)())
  \item int MyTid()
  \item int MyParentTid()
  \item void Pass()
  \item void Exit()
\end{itemize}
The kernel runs the user task defined by the assignment, and provides the requested output.

\section{Operating Instructions}
\begin{enumerate}
  \item load -b 0x00218000 -h 10.15.167.4 "ARM/ai/kernel1.elf"
  \item go
\end{enumerate}

\section{Submitted Files}
\begin{itemize}
  \item a4e2596082274858565d2979a282471f  /u/aitskovich/cs452/k1/include/bwio.h
  \item 7a3463e8a5a7fd2920efc5e1bb160b9f  /u/aitskovich/cs452/k1/include/circular\_queue.h
  \item b044b4c9d77f1d0dd9b7a2961e45a67a  /u/aitskovich/cs452/k1/include/kmemory.h
  \item 081b58b1ad6e401d9648cb88ffb30301  /u/aitskovich/cs452/k1/include/request.h
  \item 90dc03978263d9a0f904aa76fa243f1b  /u/aitskovich/cs452/k1/include/scheduling.h
  \item 7a139daaf3199b29a9fd5eda47ce189e  /u/aitskovich/cs452/k1/include/switch.h
  \item bce3e5651a8e18cc15ce7994d2353756  /u/aitskovich/cs452/k1/include/syscall.h
  \item 56e948abbb303da3d07b77c69a36f27e  /u/aitskovich/cs452/k1/include/task.h
  \item 9af226f127c1fd759530cd45236c37b8  /u/aitskovich/cs452/k1/include/ts7200.h
  \item c5a29a46e3fa5f281d5114f6ee0e6420  /u/aitskovich/cs452/k1/include/user.h
  \item db55cf26b7b39728b3f6dc490a2f3744  /u/aitskovich/cs452/k1/src/arm.mk
  \item 25757ddee6d82a16be926075e04b4972  /u/aitskovich/cs452/k1/src/bwio.c
  \item 7e851ecfa1b8faa9186a1625bd05a03b  /u/aitskovich/cs452/k1/src/circular\_queue.c
  \item 3fba02ffe8046a9db99fea23f331dc2e  /u/aitskovich/cs452/k1/src/circular\_queue\_tests.c
  \item 18f694878a418c8d04d2ac44b9380a79  /u/aitskovich/cs452/k1/src/context\_switch.s
  \item 5573ed5bb9debc960cda5dbb99176c1a  /u/aitskovich/cs452/k1/src/kernel.c
  \item e31d42639a3c6f4aa3e6bd74c7a5fd27  /u/aitskovich/cs452/k1/src/Makefile
  \item 4aa618b9753c5292e5d9e5c95d297f10  /u/aitskovich/cs452/k1/src/orex.ld
  \item cb127e770cea22148fb2f0233eaca0ca  /u/aitskovich/cs452/k1/src/scheduling.c
  \item a86e452286c5fec6e8871cd2604e28eb  /u/aitskovich/cs452/k1/src/switch.c
  \item 8f510e86e79dca63bace625c0188e3ac  /u/aitskovich/cs452/k1/src/syscall.c
  \item 102b94e6db86abbe4ba9becef5b6e7e5  /u/aitskovich/cs452/k1/src/task.c
  \item c3b46c3acfa81d47b2e01f9e1a223543  /u/aitskovich/cs452/k1/src/test.c
  \item 4cd28e770ae68bc84d735f6010d9bf6b  /u/aitskovich/cs452/k1/src/user.c
  \item 17b79aee81f0e087bb0298b74ae82aae  /u/aitskovich/cs452/k1/src/x86.mk
\end{itemize}

\section{Kernel Description}

\subsection{Tasks}

Our kernel has the concept of Tasks -- independent executions of code along with associated state. We keep track of tasks in an array of task descriptors containing

\begin{itemize}
  \item The task id
  \item The task id of the parent of the task
  \item The stack pointer for the task
  \item The saved program state register for the task
  \item The return value
  \item The task priority
  \item The stack for the task
\end{itemize}

TIDs are assigned in sequential order from 0 to 4.

\subsection{Context Switch}

When entering or exiting the kernel it's necessary to store and restore task state respectively. This is called a context switch and the code for accomplishing this is in context\_switch.s. When entering the kernel we perform the following operations:
\begin{enumerate}
  \item Switch to system mode
  \item Save registers to the user's stack
  \item Switch to supervisor mode
  \item Save the program counter to the task descriptor
  \item Switch to system state
  \item Get the stack pointer
  \item Switch to supervisor mode
  \item Save the stack pointer to the task descriptor
  \item Save the SPSR to the task descriptor
  \item Return into the main kernel loop
\end{enumerate}
When exiting the kernel we perform the following operations:
\begin{enumerate}
  \item Store kernel state onto its stack
  \item Get the stack pointer of the task from the task descriptor
  \item Get the spsr from the task descriptor
  \item Get the return value from the task descriptor
  \item Get the pc from the TD and store it in LR\_svc
  \item Switch to system mode
  \item Install the sp of the task
  \item Install the return value in r0
  \item Load the rest of the task state from its stack
  \item Switch to user mode and set the pc from LR\_svc atomically
\end{enumerate}

\subsection{System Calls}

A system call can be divided into two parts. The user mode function that uses swi to enter the kernel, and kernel mode function that performs the required operations. In our system we use a Request structure to move information between these two areas. Although we originally expected to simply pass the arguments through the registers, we realized that some system calls in later assignments used more than 4 arguments. Using more than 4 arguments requires the placement of arguments on the stack. Instead of fighting with GCC, we decided to marshal the contents of the system call into the request structure before passing it through as our only argument. The request structure encapsulates both the system call number, as well as the arguments passed in. Inside the kernel, we handle the request by switching on the system call number, and doing the proper computation afterwards. Once the result is generated, we set the return value inside the task, which is then transmitted back in the next context switch to that task.

Note that we've considered the performance implications of wrapping up the data in a Request structure. It currently vastly simplifies our code, so we would like to maintain it until we find that it's a performance problem.

\subsection{Scheduling}

For scheduling, tasks are maintained on an array of circular queues, one for each priority (4 in total). The head task on the first non-empty queue is scheduled for execution. This is not a particularly good algorithm as it wastes memory and will frequently require a linear scan of the priority queues. We're thinking of moving to using a single skip-list type queue for extremely fast scheduling. Scheduling code is found in scheduling.c.

\subsection{Bugs and Limitations}
\begin{itemize}
  \item Our system can only currently handle 5 concurrent tasks.
  \item We wrote kernel 1 with simplicity a priority over performance. Things we plan to improve in the future are:
\begin{itemize}
\item Memory layout: we're still using the memory layout from a0 with the task stacks inside the task descriptors. In the future we'll want the kernel data structures packed together nicely to improve cache performance.
\item Function calls inside context switch: we call functions within our context switch to read and write from the task descriptor struct. In the future we plan to inline these calls.
\item System call branching: we use a Request struct and a switch statement written in C to determine which system call has been issued. In the future we plan to use a jump table and leave the arguments in the correct registers for fast dispatching.
\end{itemize}
\end{itemize}

\section{Kernel Output}

Our kernel outputs the following when run:
\begin{verbatim}
Created: <1>
Created: <2>
My Tid: 3, My Parent Tid: 0
My Tid: 3, My Parent Tid: 0
Created: <3>
My Tid: 4, My Parent Tid: 0
My Tid: 4, My Parent Tid: 0
Created: <4>
First: exiting
My Tid: 1, My Parent Tid: 0
My Tid: 2, My Parent Tid: 0
My Tid: 1, My Parent Tid: 0
My Tid: 2, My Parent Tid: 0
\end{verbatim}

The user program first creates two low priority tasks 1 and 2. These two tasks don't run as we continually run the first task which has higher priority then them, until it finishes. Afterwards the Kernel creates a high priority task 3. 3's output comes before it's creation printout because it gets run instead of the first user task, until it complets since it has highest priority. A similar thing occurs with 4. Once all the high priority tasks have finished, the First task finishes and then the low priority tasks are allowed to run. Since both low priority tasks are of the same type, the alternate running (fairness along a priority level).
\end{document}
